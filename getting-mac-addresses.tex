\documentclass[12pt]{article}

\setlength{\parskip}{0.7em}

\title{Getting MAC addresses}
\author{}
\date{Last updated: \today}

\begin{document}
\maketitle

\subsection{Windows}

On Windows, you need to open a command prompt by opening the \emph{Run} dialog
and entering \texttt{cmd}. In the window that opens, type \texttt{ipconfig
/all}. You will see a dump of information about the networks and network cards
present on the computer. Sometimes this information may be too long to process
(some computers contain a lot of virtual interfaces, for example). If this is
the case, use \texttt{ipconfig /all | more} to use a pager.  Usually (but not
always) the relevant network card will be near the top. Look for something with
words like \emph{Wi-Fi}, \emph{Wireless}, or \emph{802.11} (if "Virtual" is in
the name, it probably isn't it).  If you can't find the card, ask one of us for
help.

Once you have found the correct network interface, you need to get its MAC
address. This can be acquired from the \emph{physical address} section of the
output, below the name of the interface.

\subsection{Mac OSX}

On Macs this information can be attained in a few ways. The easiest way to
explain is perhaps using the command line. Open a terminal and type
\texttt{ifconfig}. This will display information about the network interfaces
on that computer. There are a few possibilities about how this output will look
-- usually you will have \texttt{en0} and \texttt{en1}, and \texttt{en1} is the
one you want. If there is no \texttt{en1}, \texttt{en0} is probably the one you
want.

\subsection{Android}

The MAC address is visible at \emph{Settings}, \emph{About phone}, \emph{Status}
under the title \emph{Wi-Fi MAC Address}.

\subsection{iPhone and iPad}

The MAC address is visible at \emph{Settings}, \emph{General}, \emph{About}
under the title \emph{Wi-Fi Address}.

\subsection{Blackberry}

The MAC address is visible at \emph{Options}, \emph{Status} under the title
\emph{WLAN MAC address}.

\subsection{Nokia phones}

The MAC address is generally visible upon dialling the number
\texttt{*\#62209526\#} (\texttt{*\#wlan0mac\#}).

I have also encountered some Nokia phones (notably, the Nokia Asha) where you
have to go into the wireless networks menu, highlight an access point, press
the menu button, and select \emph{Details}, which reveals the MAC address.

\end{document}
