\documentclass[12pt]{article}

\setlength{\parskip}{0.7em}

\title{Cauxpublic}
\author{}
\date{Last updated: \today}

\begin{document}
\maketitle

\section{Connecting to cauxpublic}

\subsection{Windows}
Press \texttt{Ctrl-r} to open a \emph{Run} dialog, and then enter \texttt{\textbackslash\textbackslash
icserver\textbackslash{}cauxpublic}. This should open a new explorer window
with the contents of cauxpublic listed.

\subsection{Mac}
You can do this by clicking on the desktop or a
Finder window and pressing \texttt{Command-k}. Once there, connect to the server by
entering \emph{icserver/cauxpublic}. Select \emph{Guest}.

\section{Creating a desktop shortcut to cauxpublic}

\subsection{Windows}
Right click on the desktop and select \emph{New}, and then \emph{Shortcut}.
A dialog box will appear asking for a location. Enter \texttt{\textbackslash\textbackslash
icserver\textbackslash{}cauxpublic}. Click \emph{OK}, and then \emph{OK} again,
and a shortcut will be created on the desktop.

\subsection{Mac}
Connect to the cauxpublic share as described in the \emph{Connecting to cauxpublic}
section. Then, once the window is open, hold \texttt{Command-alt} and drag the
current year's directory to the desktop. This will create a link on the desktop
to that directory. Once it is there, rename it to \emph{cauxpublic}.

\end{document}
